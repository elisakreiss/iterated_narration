
% 
% Annual Cognitive Science Conference
% Sample LaTeX Paper -- Proceedings Format
% 

% Original : Ashwin Ram (ashwin@cc.gatech.edu)       04/01/1994
% Modified : Johanna Moore (jmoore@cs.pitt.edu)      03/17/1995
% Modified : David Noelle (noelle@ucsd.edu)          03/15/1996
% Modified : Pat Langley (langley@cs.stanford.edu)   01/26/1997
% Latex2e corrections by Ramin Charles Nakisa        01/28/1997 
% Modified : Tina Eliassi-Rad (eliassi@cs.wisc.edu)  01/31/1998
% Modified : Trisha Yannuzzi (trisha@ircs.upenn.edu) 12/28/1999 (in process)
% Modified : Mary Ellen Foster (M.E.Foster@ed.ac.uk) 12/11/2000
% Modified : Ken Forbus                              01/23/2004
% Modified : Eli M. Silk (esilk@pitt.edu)            05/24/2005
% Modified : Niels Taatgen (taatgen@cmu.edu)         10/24/2006
% Modified : David Noelle (dnoelle@ucmerced.edu)     11/19/2014
% Modified : Elisa Kreiss (ekreiss@uos.de)    10/10/2016

%% Change ''letterpaper'' in the following line to ''a4paper'' if you must.

\documentclass[10pt,letterpaper]{article}

\usepackage{cogsci}
\usepackage{pslatex}
\usepackage{apacite}
\usepackage{amsmath,amssymb}
\usepackage{graphicx}
\usepackage{color}
\usepackage{url}
\usepackage{todonotes}
\usepackage{mathtools}
\usepackage{stmaryrd}
\usepackage{booktabs}
\usepackage{array}

\newcommand{\den}[2][]{
\(
\left\llbracket\;\text{#2}\;\right\rrbracket^{#1}
\)
}

%\newcommand{\url}[1]{$#1$}


\definecolor{Blue}{RGB}{0,0,255}
\definecolor{Green}{RGB}{10,200,100}
\definecolor{Red}{RGB}{255,0,0}
\definecolor{Orange}{RGB}{255,140,0}
\newcommand{\jd}[1]{\textcolor{Blue}{[jd: #1]}}  
\newcommand{\rdh}[1]{\textcolor{Red}{[rdh: #1]}}  
\newcommand{\red}[1]{\textcolor{Red}{#1}}
\newcommand{\ndg}[1]{\textcolor{Green}{[ndg: #1]}}
\newcommand{\ek}[1]{\textcolor{Orange}{[ek: #1]}} 

 \newcommand{\denote}[1]{\mbox{ $[\![ #1 ]\!]$}}


\newcommand{\subsubsubsection}[1]{{\em #1}}
\newcommand{\eref}[1]{(\ref{#1})}
\newcommand{\tableref}[1]{Table \ref{#1}}
\newcommand{\figref}[1]{Fig.~\ref{#1}}
\newcommand{\appref}[1]{Appendix \ref{#1}}
\newcommand{\sectionref}[1]{Section \ref{#1}}

\title{Influences of Pragmatics on Iterated Narration}

 
\author{{\large \bf Elisa Kreiss, Judith Degen, Michael Franke} \\
  ekreiss@uos.de, jdegen@stanford.edu, mfranke@uos.de\\
  Department of Psychology, STREET \\
  Osnabrueck, Germany}



\begin{document}

\maketitle


\begin{abstract}

\textbf{Keywords:} 
keywords
\end{abstract}

\section{Introduction}

\section{Experiment}

\subsection{Methods}

\subsubsection{Participants}
We recruited XX participants over Amazon's Mechanical Turk. 
\ek{say something about payment and duration}\\
Each participant was given a cover story, i.e., a situation they were to imagine themselves in. After the participants were introduced to the context, they were presented with a story that they were then asked to retell. The participants were instructed not to take notes during the reading phase. Between reading and recollection, we asked whether they had already heard the story before. 

\subsubsection{Materials \& Design}
Before reading the story, each participant was presented with one out of three cover stories. In that way, participants were provided with a purpose for the following task. One cover story instructed the participants to imagine themselves to be an apprentice in a lawyer's office and it is their task to brief their boss about current or potential cases. The second cover story sets the scene at a friend's party. That friend calls out a competition of "Who tells the best story" on the basis of actual news stories. The third cover story functions as a neutral condition in which no context is given. The participants are only informed that they will be given a story and are supposed to recollect it later. This instruction is similar to the one used in \cite{Bartlett:1932}.\\
The presented stories are inspired by actual news reports from the last year. In general, we chose a neutral writing style, but also used peculiar vocabulary and phrases from the original articles. The topics are widespread and involve, e.g., trespassing, smuggling and destruction of property.

\subsection{Procedure}

\subsection{Annotation}

\subsection{Results and discussion}

\section{Discussion and conclusion}

\section{Acknowledgments}
\small



\bibliographystyle{apacite}

\setlength{\bibleftmargin}{.125in}
\setlength{\bibindent}{-\bibleftmargin}

\bibliography{refs}


\end{document}